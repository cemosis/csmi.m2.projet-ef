\feelchapter{2D Maxwell simulation in a diode}
            {2D Maxwell simulation in a diode}
            {Thomas Strub, Philippe Helluy, Christophe Prud'homme}
            {cha:maxwell-2d}
\newcommand{\E}{\ensuremath{{\bm E}}\xspace}
\newcommand{\B}{\ensuremath{{\bm B}}\xspace}
\newcommand{\J}{\ensuremath{{\bm J}}\xspace}
\section{Description}
\label{sec:description}

The Maxwell equations read:


\begin{eqnarray*}
\frac{-1}{c^{2}}\frac{\partial \E}{\partial t}+\nabla\times \B & = & \mu_{0} \J\\
\B_{t}+\nabla\times \E & = & 0\\
\nabla \cdot \B & = & 0\\
\nabla \cdot \E & = & \frac{\rho}{\epsilon_{o}}
\end{eqnarray*}

where \E is the electric field, \B the magnetic field,
\J the current density, $ c $ the speed of light, $ \ rho $
density of electric charge, $ \ mu_ {0} $ the vacuum permeability
and $ \ epsilon_ {0} $ the vacuum permittivity.

In the midst industrial notament in aeronautics, systems
Products must verify certain standards such as the receipt
an electromagnetic wave emitted by a radar does not cause
the inefficassité of part or all of the hardware in the
system.

Thus, the simulation of such situations can develop when
or during the certification of a new product to test its reaction
to such attacks.

Also note that the last two equations are actually
initial conditions, since if we assume they are true at the moment
$ t = 0$ then it can be deduced from the first two.

At $t=0s$, we suppose that
\begin{eqnarray}
\nabla \cdot \B & = & 0\\
\nabla \cdot \E & = & \frac{\rho}{\epsilon_{o}}
\end{eqnarray}

Suppose that $\B = (B_x, B_y, B_z )^T$ and $\E=(E_x,E_y,E_z)^T$
i.e.\begin{eqnarray}
\frac{\partial B_{x}}{\partial x}(t=0)+\frac{\partial B_{y}}{\partial y}(t=0)+\frac{\partial B_{z}}{\partial z} & (t=0)= & 0\\
\frac{\partial E_{x}}{\partial x}(t=0)+\frac{\partial E_{y}}{\partial
y}(t=0)+\frac{\partial E_{z}}{\partial z} & (t=0)= & \frac{\rho}{\epsilon_{o}}
\end{eqnarray}


Differentiating the first of these two equations with respect to time,
we get:


\begin{multline}
  \label{eq:6}
\frac{\partial}{\partial t}\frac{\partial}{\partial
x}B_{x}+\frac{\partial}{\partial t}\frac{\partial}{\partial y}B_{y}+\frac{\partial}{\partial t}\frac{\partial}{\partial z}B_{z} = \\
\frac{\partial}{\partial x}\left(\frac{\partial}{\partial y}E_{z}-\frac{\partial}{\partial z}E_{y}\right)+\frac{\partial}{\partial y}\left(\frac{\partial}{\partial z}E_{x}-\frac{\partial}{\partial x}E_{z}\right)+\frac{\partial}{\partial z}\left(\frac{\partial}{\partial x}E_{y}-\frac{\partial}{\partial y}E_{x}\right)\\
 =  0
\end{multline}



thanks to

\begin{equation}
  \label{eq:3}
  \B_{t}+\nabla\times \E=0
\end{equation}



So, for all $t\geq0$,
\begin{equation}
  \label{eq:4}
  \nabla \cdot \B(t)=\nabla\cdot\B(0)=0
\end{equation}

We deduce the same way the second equation, using the charge
conservation equation :


\begin{equation}
  \label{eq:2}
  \frac{\partial \rho}{\partial t} + \nabla \cdot  (\rho \J) = 0
\end{equation}



\section{Variational formulation}
\label{sec:vari-form}

\section{Implementation}
\label{sec:implementation}

\section{Numerical Results}
\label{sec:numerical-results}


%%% Local Variables:
%%% coding: utf-8
%%% mode: latex
%%% TeX-PDF-mode: t
%%% TeX-parse-self: t
%%% x-symbol-8bits: nil
%%% TeX-auto-regexp-list: TeX-auto-full-regexp-list
%%% TeX-master: "../feel-manual"
%%% ispell-local-dictionary: "american"
%%% End:
