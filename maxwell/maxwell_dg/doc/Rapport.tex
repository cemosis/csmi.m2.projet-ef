\documentclass[a4paper,oneside,10pt]{report}
\usepackage[utf8]{inputenc}
\usepackage[francais]{babel}
\usepackage{fancyhdr}
\usepackage[nodisplayskipstretch]{setspace} 
\usepackage[dvips,a4paper]{geometry}
\usepackage{titlesec} 
\usepackage[hang,small,bf]{caption}
\usepackage{footmisc}
\usepackage{color}
\usepackage{xcolor}
\usepackage{listings}
\usepackage{caption}
\usepackage[caption=false]{subfig} 
\usepackage{graphicx}
\usepackage{capt-of}
\usepackage{amsmath,amssymb,amsfonts}
\usepackage{amsthm}
\usepackage{float}
\usepackage[colorlinks=true,linkcolor=black,bookmarkstype=toc,linktocpage=true]{hyperref}
\setstretch{1.2}
\geometry{lmargin=2.5cm,rmargin=2.5cm,vmargin=4cm}
\setlength{\parindent}{1cm}
%\renewcommand{\thechapter}{\Roman{chapter}}
\titleformat{\chapter}[display]
{\bfseries\Large}
{\filleft\MakeUppercase{\chaptertitlename} \Huge\thechapter}
{4ex}
{\titlerule
\vspace{2ex}%
\filright}
[\vspace{2ex}%
\titlerule]
\renewcommand*\thepart{\arabic{part}}%
\titlespacing*{\chapter}{0pt}{-60pt}{40pt}
\pagestyle{plain} 
\fancyhf{} 
\setlength{\footskip}{30pt} 
\setlength{\textheight}{646pt}
\setlength{\hoffset}{0pt}
\cfoot{\thepage} 
\renewcommand{\headrulewidth}{1.2pt}
\lhead{\leftmark}
\setlength{\skip\footins}{1cm}
\setlength{\footnotesep}{0.3cm}
\renewcommand{\footnotelayout}{\scriptsize}
\renewcommand{\chaptermark}[1]{\markboth{\footnotesize{#1}}{}}
\setstretch{1.2}
\pagenumbering{arabic}
\renewcommand\lstlistingname{Source}
\renewcommand\lstlistlistingname{Source}
\lstset{
language=fortran,
 basicstyle=\setstretch{1}\footnotesize,
 numbers=left,
 numberstyle=\scriptsize,
 numbersep=4pt,
 tabsize=2, 
 extendedchars=true,
 breaklines=true,
 % keywordstyle=\bfseries{\color{black}},
  commentstyle=\color{gray},
  frame=b,
  stringstyle=\setstretch{1}\ttfamily,
  showspaces=false,
  showtabs=false,
  xleftmargin=17pt,
  framexleftmargin=17pt,
  framexrightmargin=5pt,
  framexbottommargin=4pt,
  showstringspaces=false,
		 }
 \DeclareCaptionFormat{listing}{\colorbox[gray]{0.20}{\parbox{\textwidth}{\hspace {15pt}#1#2#3}}}
% \captionsetup[lstlisting]{format=listing,labelfont=white,textfont=white,
% singlelinecheck=false,margin=0pt,font={bf,footnotesize}}
\setlength{\parindent}{0cm}

\DeclareCaptionFont{white}{\color{white}}
%\DeclareCaptionFormat{listing}{\colorbox[cmyk]{0.43, 0.35, 0.35,0.01}{\parbox{\textwidth}{\hspace{15pt}#1#2#3}}}
\captionsetup[lstlisting]{format=listing,labelfont=white,textfont=white, singlelinecheck=false, margin=0pt, font={bf,footnotesize}}

\newtheorem{mydef}{Définition}

\newcommand\Image[3][]{%
  \tabular[b]{@{}c@{}}\includegraphics[#1]{#2}\\
    #3
  \endtabular}


% \newcommand{\E}{\mathbf{E}}
% \newcommand{\H}{\mathbf{H}}


\begin{document}
\def\chaptername{travaux pratiques} 

\begin{titlepage}
\setstretch{3}
\newcommand{\HRule}{\rule{\linewidth}{0.5mm}}

\center
 
\textsc{\LARGE université de Strasbourg}\\[1.5cm]
\textsc{\Large ufr mathématiques informatique }\\[0.5cm]  
\textsc{\Large master ii - csmi}

\HRule \\

{ \textsc{\huge \bfseries Projet Volumes Finis\\
Résolution des équations de Maxwell 2D-TM par méthode RKDG via la librairie \texttt{Feel++}}}
\HRule \\
\vfill
\textbf{Pierre \textsc{GERHARD}}
\vfill
\textbf{Sous la direction de}\\
\textbf{Christophe \textsc{PRUD'HOMME}}.
\vfill
{\large Janvier 2015}\\[0cm]

\end{titlepage}
 
\pagestyle{plain}
\setstretch{1.2}


\tableofcontents

\chapter{Modèle mathématique}
\section{Présentation du problème}
On considère ici les équations de Maxwell 2D a-dimensionnées et sans terme source telles que $\forall (\mathbf{x},t) \in \Omega \times [0,T]$,


\begin{equation}
\begin{aligned}
\label{eq:1}
\partial_t \mathbf{E} + \nabla \times \mathbf{H} &= 0,\\
\partial_t \mathbf{H} + \nabla \times \mathbf{E} &= 0.
\end{aligned}
\end{equation}

où $\mathbf{E}(\mathbf{x},t)$, désigne le champ électrique et $\mathbf{H}(\mathbf{x},t)$ le champ magnétique, tout deux dépendant de la variable d'espace $\mathbf{x}=(x_1,x_2)\in \Omega \subset \mathbb{R}^2$ et du temps $t$. En se plaçant en coordonnées cartésiennes et considérant le mode transverse magnétique (TM) (où $\mathbf{H}$ à une unique composante suivant $z$), il vient alors,

\begin{eqnarray}
\partial_t E_x - \partial_y H_z  &=& 0, \label{eq:2a}\\
\partial_t E_y + \partial_x H_z &=& 0, \label{eq:2b}\\
\partial_t H_z + \partial_x E_y - \partial_y E_x &=& 0, \label{eq:2c}
\end{eqnarray}


Ce sont ces équations que nous allons chercher à résoudre numériquement. Pour cela on utilisera la librairie C \texttt{SCHNAPS}. Cette librairie permet la résolution par méthode Galerkin discontinue de systèmes de lois de conservation hyperboliques de forme générale,

\begin{equation}
\label{eq:sysh}
\partial_t \mathbf{W} +  \partial_i F^i(\mathbf{W}) = S(\mathbf{W}).
\end{equation}

où, $\mathbf{W}(\mathbf{x},t) \in \mathbb{R}^n$ désigne la variable conservative dépendant de la variable d'espace $\mathbf{x} = (x_1,x_2,x_3)$ et du temps $t$. Les termes $F^i$ désignant quant à eux les composantes du flux et $S$ le terme source.\newline

\textbf{Remarque } : Afin d'alléger la notation, on utilise ici la convention d'Einstein sur les indices répétés : on substituera par $\partial_i F^i$ le terme $\sum_{i=1}^{3}\partial _i F^i$.\newline


\section{Système de lois de conservation}

Afin de résoudre numériquement les équations (\ref{eq:2a}), (\ref{eq:2b}) et (\ref{eq:2c}) via la librairie \texttt{SCHNAPS}, il nous faut tout d'abord les réécrire sous la forme d'un système de lois de conservation du type (\ref{eq:sysh}).\newline

En notant tout d'abord le vecteur $\mathbf{W}$ et les matrices, $\mathbf{A}^1$ et $\mathbf{A}^2$, tels que

\begin{equation}
\mathbf{W}=
\begin{pmatrix}
E_x\\
E_y\\
B_z
\end{pmatrix},
\qquad
\mathbf{A}^1=
\begin{pmatrix}
0 & 0 & 0\\
0 & 0 & 1\\
0 & 1 & 0
\end{pmatrix},
\qquad
\mathbf{A}^2=
\begin{pmatrix}
0 & 0 & -1\\
0 & 0 & 0\\
-1 & 0 & 0\\
\end{pmatrix},
\end{equation}
les équations (\ref{eq:2a}), (\ref{eq:2b}) et (\ref{eq:2c}) deviennent,

\begin{equation}
\label{eq:sysh2}
\partial_t \mathbf{W} +  \mathbf{A}^i\partial_i \mathbf{W} = 0 \quad \forall (\mathbf{x},t) \in \Omega \times [0,T],
\end{equation}


Montrons à présent l'hyperbolicité du système (\ref{eq:sysh2}), grâce au théorème suivant, (\ref{})
\newline
\begin{mydef}
Le système (\ref{eq:sysh2}) est dit hyperbolique si pour tout vecteur $(n_1,n_2 )^T \in 
\mathbb{R}^2$, la matrice $\mathbf{A}^in_i = \mathbf{A}^1 n_1 + \mathbf{A}^2 n_2$ est diagonalisable et ses valeurs propres sont réelles.\newline
\end{mydef}

Cela nous donne, $\forall (n_1,n_2 )^T \in \mathbb{R}^2$,

\begin{equation}
\mathbf{A}^i n_i = \mathbf{A}^1 n_1 + \mathbf{A}^2 n_2 =
\begin{pmatrix}
0 & 0 & 0\\
0 & 0 & n_1\\
0 & n_1 & 0
\end{pmatrix}
+
\begin{pmatrix}
0 & 0 & -n_2\\
0 & 0 & 0\\
-n_2 & 0 & 0
\end{pmatrix}
=
\begin{pmatrix}
0 & 0 & -n_2\\
0 & 0 & n_1\\
-n_2 & n_1 & 0\\
\end{pmatrix}
\end{equation}


En résolvant en $\{\lambda_1,\lambda_2,\lambda_2\}$ l'équation suivante,

\begin{equation}
\mbox{det} (\mathbf{A}^i n_i - \lambda\mathbb{I}) =
\begin{vmatrix}
-\lambda_1 & 0 & -n_2\\
0 & -\lambda_2 & n_1\\
-n_2 & n_1 & -\lambda_3
\end{vmatrix}
=0,
\end{equation}

on obtient comme valeurs propres de la matrice $\mathbf{A}^i n_i$, 
\begin{equation}
\label{eq:vp}
\lambda_1 = -\sqrt{n_1^2 + n_2^2},\quad \lambda_2 =0,\quad \lambda_3 = \sqrt{n_1^2 + n_2^2},
\end{equation}

et en posant pour tout $(n_1^2 + n_2^2) \neq 0 $ 
$$r=\sqrt{n_1^2 + n_2^2},$$
 on a pour vecteur propre associés,

\begin{equation}
\mathbf{v}_{\lambda_1}=\frac{1}{r}
\begin{pmatrix}
n2\\
-n1\\
r
\end{pmatrix},
\qquad
\mathbf{v}_{\lambda_2}=\frac{1}{r}
\begin{pmatrix}
-\frac{rn_1}{n_2}\\
r\\
0
\end{pmatrix},
\qquad
\mathbf{v}_{\lambda_3}=\frac{1}{r}
\begin{pmatrix}
-n2\\
n1\\
r
\end{pmatrix}.
\end{equation}

Ainsi, le système est bien hyperbolique. De plus, comme la matrice $\mathbf{A}^in_i $ est symétrique, le système sera dit de Friedrich.\\

Afin d'assurer l'existence et l'unicité d'une solution au problème (\ref{eq:sysh2}) nous lui ajoutons la condition initiale suivante,

\begin{equation}
\label{eq:ini} 
\mathbf{W}(\mathbf{x},0)=\mathbf{W}_0(\mathbf{x}),
\end{equation}

ainsi que la condition au bord,

\begin{equation}
\label{eq:bound} 
\mathbf{M}(\mathbf{x})(\mathbf{W}-\mathbf{W}_{in})=0\quad\forall \mathbf{x} \in \partial \Omega.
\end{equation}

avec, $\mathbf{W}_0$, $\mathbf{W}_{in}$ et $\mathbf{M}$ donnés.\newline

L'existence et l'unicité d'une solution à (\ref{eq:sysh2}) sont subordonnées au fait que la matrice $\mathbf{M}$ vérifie,

\begin{equation}
\label{eq:ctr1}
\mathbf{N} = \frac{1}{2} \mathbf{A}^in_i + \mathbf{M}\quad \mbox{(existence)},
\end{equation}

avec $\mathbf{N}$ une matrice positive et que,

\begin{equation}
\label{eq:ctr2}
\mbox{Ker}\mathbf{M} \subset \mbox{Ker}\mathbf{A}^in_i \quad \mbox{(unicité)}.
\end{equation}

\textbf{Note :} Les contraintes (\ref{eq:ctr1}), (\ref{eq:ctr2}) portant sur la matrice $\mathbf{M}$ s'obtiennent par calcul d'une estimation d'énergie vérifiant le théorème de Hille-Yosida. Nous ne détaillerons pas ici ce calcul. Pour plus de détails on renverra le lecteur à (\ref{}).\\
\newline


Ainsi sous ces conditions, le problème revient alors à trouver $\mathbf{W}\in\Omega \times [0,T]$ solution de l'équation,

\begin{equation}
\label{eq:final}
\mathcal{P} \left\{
\begin{aligned}
\partial_t \mathbf{W} +  \mathbf{A}^i\partial_i \mathbf{W} &= 0 &\quad &\forall (\mathbf{x},t) \in \Omega \times [0,T],\\
\mathbf{W}(\mathbf{x},0)&=\mathbf{W}_0(\mathbf{x}) &\quad &\forall \mathbf{x} \in \Omega,\\
\mathbf{M}(\mathbf{x})(\mathbf{W}-\mathbf{W}_{in})&=0&\quad&\forall \mathbf{x} \in \partial\Omega.
\end{aligned}
\right.
\end{equation}


\section{Formulation faible du problème}

On considère un maillage fait de d'un nombre finis $N\in \mathbb{N}$ d'ouverts disjoints (ou cellules) $\Omega_k \subset \Omega$, $k=1\dots N$ tel que,
\begin{equation}
\bar{\Omega} = \cup_k \bar{\Omega}_k.
\end{equation}

Nous allons à présent raisonner localement, en définissant l'ensemble $V$ tel que la restriction de $\mathbf{W}(x,t)$ sur la maille $\Omega_k$ appartienne à $(H^1[\Omega_k])^3$, d'où,




\section{Flux numérique}
Comme $\mathbf{W}$ n'est pas supposé continue aux interfaces $\partial \Omega_k$ il nous est interdit de définir $F(\mathbf{W},n)$
Pour les équations de Maxwell on utilisera un flux numérique décentré en amont tel que,

%general
% \begin{equation}
% \begin{aligned}
% F^*(\mathbf{W}_L,\mathbf{W},\mathbf{n}) &= \frac{1}{2}\mathbf{A}^i n_i(\mathbf{W_L} +\mathbf{W_R}) - \frac{1}{2}|\mathbf{A}^i n_i|(\mathbf{W_R} - \mathbf{W_L}),\\
% &=\frac{1}{2}\mathbf{A}^i n_i(\mathbf{W_L} +\mathbf{W_R}) - \frac{1}{2}(\mathbf{A}^i n_i^+ - \mathbf{A}^i n_i^-)(\mathbf{W_R} - \mathbf{W_L}),
% \end{aligned}
% \end{equation}




\begin{mydef}
Pour les équation de Maxwell le flux numérique décentré en amont est donnée par,
\begin{equation}
\label{eq:fluxnum}
F^*(\mathbf{W_L},\mathbf{W_R},\mathbf{n}) = \mathbf{A}^i n_i^+\mathbf{W_L} + \mathbf{A}^i n_i^-\mathbf{W_R}
\end{equation}
avec $\mathbf{A}^i n_i^+$ et $\mathbf{A}^i n_i^-$ les parties positives et négatives de $\mathbf{A}^i n_i$.
\end{mydef}

Pour les obtenir $\mathbf{A}^i n_i^+$ et $\mathbf{A}^i n_i^-$ on utilise tout d'abord le fait que, 

\begin{equation}
\mathbf{D} =\mathbf{P}^{-1}\mathbf{A}^in_i\mathbf{P}=
\begin{pmatrix}
-r & 0 &0 \\
0& 0 & 0\\
0 & 0 & r\\
\end{pmatrix},
\end{equation}

avec $\mathbf{P}$ la matrice de passage et $\mathbf{P}^{-1}$ son inverse, telles que

\begin{equation}
\mathbf{P}=\frac{1}{r}
\begin{pmatrix}
n2 & -\frac{rn_1}{n_2} &-n2\\
-n1 & r & n1\\
r & 0 & r\\
\end{pmatrix}
\quad \mbox{et} \quad
\mathbf{P}^{-1}=\frac{1}{2r}
\begin{pmatrix}
n_2 & -n_1 &r\\
\frac{2n_1n_2}{r} & \frac{2n_2^2}{r} & 0\\
-n_2^2 & n_1 & r\\
\end{pmatrix}.
\end{equation}

Ensuite en notant par $\mathbf{D}^+$ et $\mathbf{D}^+$ les parties positive et négative de $\mathbf{D}$, il vient,

\begin{equation}
\mathbf{A}^in_i^+= \mathbf{P}\mathbf{D}^+\mathbf{P}^{-1} = \frac{1}{2}
\begin{pmatrix}
\frac{n_2^2}{r} & \frac{-n_1n_2}{r}  & -n_2 \\
-n_1n_2& n_1^2 & n_1\\
-n_2& n_1 & r\\
\end{pmatrix}
\end{equation}
et
\begin{equation}
\mathbf{A}^in_i^-= \mathbf{P}\mathbf{D}^-\mathbf{P}^{-1} = \frac{1}{2}
\begin{pmatrix}
\frac{-n_2^2}{r} & \frac{n_1n_2}{r}  & -n_2 \\
n_1n_2& -n_1^2 & n_1\\
-n_2& n_1 & -r\\
\end{pmatrix}.
\end{equation}



\bibliographystyle{plain}
\bibliography{Rapport}

\end{document}
