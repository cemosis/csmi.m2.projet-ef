\documentclass[12pt]{article}
\usepackage{amsmath}
\usepackage{amsfonts}
\usepackage{amssymb}
\usepackage{amsthm}
\usepackage[english]{babel}
\usepackage[babel,kerning=true]{microtype}
\usepackage[nottoc,notlof,notlot]{tocbibind}
\usepackage[T1]{fontenc}
\usepackage{graphicx}
\usepackage{caption}
\usepackage{subcaption}
\usepackage{xcolor}
\usepackage{bbm} 
\usepackage{listings}
\usepackage{pgfplots}
\usepackage{hyperref}
\hypersetup{
    colorlinks=true,
    linkcolor=blue,
    urlcolor=red,
    linktoc=all
    }
\usepackage{pgfplotstable}
\usepackage{enumerate}
\usepackage{enumitem}
\usepackage[utf8x]{inputenc}
\usepackage{listings}
\usepackage{geometry}
\usepackage{tikz}
\graphicspath{{picture/}}
\geometry {
    left  = 2.5cm,
          right  = 2.5cm,
          bottom = 2.5cm ,
          top = 2.5cm ,
}
\pgfplotsset{scaled ticks=false,compat=1.8}
\lstdefinestyle{customc++}{
    belowcaptionskip=1\baselineskip,
        breaklines=true,
        frame=single,
        %linewidth=7.5cm,
        framexleftmargin=5mm,
        %frameround=tttt,
        %framexrightmargin=5mm,
        xleftmargin=\parindent,
        language=C++,
        showstringspaces=false,
        basicstyle=\footnotesize\ttfamily,
        keywordstyle=\color{green!40!black},
        ndkeywordstyle=\color{orange},
        commentstyle=\color{purple!40!black},
        identifierstyle=\color{blue},
        stringstyle=\color{red},
        numbers=left,
        numbersep=7pt,
}

%\frenchbsetup{StandardLists=true}
\lstset{style=customc++, emph={int,double,void, Double}, emphstyle=\color{red}, emph={[2]wavJava, Spectrum}, emphstyle=[2]\color{orange}}
\title{Agile Software Method}
\author{ 
  Mamadou Camara ,\\
  Cemosis,\\
  UFR Mathématiques et Informatique/UDS}

        \date{\today}
\newpage

        \begin{document}
        \maketitle
        \tableofcontents
\newpage
\section{Introduction}
Agile Software Development is a method of web management project.
It's firstly introtuced in 2001, with publication of ``Manifesto for Agile Softeware Development''.
The aim of authors is to offer new mode computer development.

The method works in base of iterative and incremental,doing jobs step by step.
Collaborators work by priority, we have also control phases and exchange with client. 
It's a flexible method which help to detect bug in need of regular feed-back with client and adapt to his needs.
\section{Core Values of Agile Approach}
 Essentially, we have four values of Agile method:
\begin{itemize}
\item Communication: exchange with collaborators is primordial,it is the thing to do before any jobs.
\item Change: The provider should welcome any change and modifying in project process.
\item Collaboration: relationships  with client are not based on contract but on will of differents parts to meet the precise needs of client.
\item Operational functions: delivred to clients operational product.
\end{itemize}
\section{Principles of Agile Method}
\begin{itemize}

 \item First is to satisfy client,and see eventual changes as concurential advantages.
\item We may often provide client an operational product (at most two weeks).
\item Constant collaboration with work team and clients.
\item Need motivated and welded persons to do project.
\item  Promoted face to face exchanges,it is simple and easy.It is more effective.
\item  Project progress can be measured through an operational softaware.
\item The project must be built around a sustainable and steady pace.
\item Pay special attention to technical excellence and good design.
\item Promote simplicity by minimizing the workload without interest.
\item Encourage self organized teams that produce a better design, architecture and specification.
\item Constant reflection of the team to improve its efficiency by regularly adjusting its behavior.
\end{itemize}
\section{Conclusion}
Agile Method is a paradigm which help for good management of a software project.
It provides each collaborator bring added-values in project.
This method is applied in others domains such that business management.It is called there business management process. 
\end{document}
