\documentclass[paper=a4, fontsize=11pt]{scrartcl}

\usepackage[T1]{fontenc}
\usepackage{fourier}
\usepackage[english]{babel}

\usepackage{sectsty}
\allsectionsfont{\centering \normalfont\scshape} 

\usepackage{fancyhdr}
\pagestyle{fancyplain}
\fancyhead{}
\fancyfoot[L]{}
\fancyfoot[C]{}
\fancyfoot[R]{\thepage}
\renewcommand{\headrulewidth}{0pt}
\renewcommand{\footrulewidth}{0pt}
\setlength{\headheight}{13.6pt}

\setlength\parindent{0pt}

\newcommand{\horrule}[1]{\rule{\linewidth}{#1}}

\title{	
\normalfont \normalsize 
\textsc{university of strasbourg} \\ [25pt]
\horrule{0.5pt} \\[0.4cm]
\huge Minimalist Software Development \\
\horrule{2pt} \\[0.5cm]
}

\author{Djebril Mokaddem}

\date{\normalsize\today}

\begin{document}

\maketitle

\section{Minimalist report}

This method is mainly based on Pareto's Law. Basically, the latter states that 80\% of the results come from 20\% of effort. Therefore, we can say that it's a non perfectionist approach and seeks to focus first in the essentials points (or core) of the specifications. As the outcome keeps being polished throughout several development passes, the developer tries to improve his code to make it simpler and cleaner. This implies he has to heavily make use of his creativity, because simpler is harder than complex. Moreover, this approach allows software development to be more flexible. Indeed, instead of dealing with complex code (having a lot of dependancies and/or APIs), the developer won't be afraid of starting all over again or undoing something that is no longer needed in the specs. In addition to that, his programming skills would improve significantly faster than the average since he will be able to test rapidly different kind of configurations.\newline

To wrap this up, I would say that Minimalism requires a lot of creativity in order to find new ways of doing more with less (minimum of technical skills). It allows developers to be more innovative in programming, and encourages them to sharpen their way of thinking as entrepreneurial-minded persons.

\end{document}
